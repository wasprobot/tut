\documentclass{article}
\usepackage{amsmath,amssymb}
\newcounter{question}
\setcounter{question}{0}
\begin{document}

\newcommand\Que[1]{%
    \leavevmode\par
    \stepcounter{question}
    \noindent
    Q.\thequestion #1\par}

\newcommand\Ans[2][]{%
    \leavevmode\par\noindent
    {A) \textbf{#1}#2\par}}

\Que{
    Prove by induction: $1+2+3+\dots+n = \frac{n(n+1)}{2}$

    For \textbf{all} n.
}
\Ans{
    If we can prove that if the above statement is true for 
    a given $n$,\\
    it is also true for $n+1$. We would have
    proven it for \textbf{all} numbers!\\

    \textit{basis step}: Let $n=1$ The sum $1+\dots+1 = 1 = \frac{1(2)}{2}$\\

    \textit{induction step}: Assume that the statement is
    true for a given $k$.\\
    This means $1+2+3+\dots+k=\frac{k(k+1)}{2}$.\\
    Now we prove this to be true for $k+1$.\\

    Let's consider $1+2+3+\dots+(k+1)$\\

    \begin{align*}
        1+2+3+\dots+(k+1) 
        & = 1+2+3+\dots+k+(k+1)\\
        & = \frac{k(k+1)}{2} + (k+1)\\
        & = \frac{k(k+1)}{2} + \frac{2(k+1)}{2}\\
        & = \frac{(k+1)(k+2)}{2}\\
        & = \frac{(k+1)((k+1)+1)}{2}\\
    \end{align*}

}
\end{document}